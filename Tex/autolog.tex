\documentclass[10pt,a4paper]{article}
\pagestyle{plain}
\textwidth 16cm
\textheight 21cm
\oddsidemargin -0.5cm
\topmargin 0cm

\parindent0pt
\parskip10pt

\title{Autologger}
\author{Robert J. Smith.}
\date{}
\begin{document}
\pagenumbering{arabic}
\thispagestyle{empty}
\maketitle

\bigskip
\bigskip
\tableofcontents
%\listoffigures
%\listoftables
\newpage


\section{Purpose of Software}

This software is to be run first thing in the morning after the Dp(RT)
to output a text file list of all the exposures taken. Little detail 
is given.  It is intended to contain all the basic observational 
parameters found in a traditional hand written observing log. It also 
contains some fault logging which should help in identifying corrupt 
files or those that failed the automated pipeline reductions.
Files are written to the log in the order in which the exposures
were made.

By running after dprt we can include sky brightness, extinction estimates 
and a possible bad data flag. The {\tt autolog} will preferentially use 
files which according to their filenames have passed through Dp(RT),
but if no reduced file is available, it will use an unreduced file 
instead if it is present.

The future possibility of dredging the system logs for events such as 
dome opening and closing is worth remembering, but is not currently
implimented in any manner.

\section{Execution Parameters}
The executable {\tt autolog} takes a single command line parameter,
being the path to a directory containing all the files to logged.


\section{Summary of Program Flow}
The procedures executed by {\tt autolog} are listed below in order. 
Hopefully re-reading this will make it easier to update or modify the
code when we come back to it later!

\begin{itemize}
\item The named directory is opened
\item A progress / error log called `autologger.log' is opened in the working directory. This will contain any error messages generated by {\tt autolog}. 
\item Each file in the directory is inspected. No further action is taken 
for any file which does not have the `.fits' extension.
\item If the `p' flag in the filename (see `Liverpool Telescope 
Fits Keyword Specification') indicates this file has not been reduced, 
the directory is scanned to see if a reduced one is available. If it is,
the unreduced file is abandoned. If no reduced file is available, 
{\tt autolog} will continue and do the best it can with this file.
\item The {\tt LogInfo\_vec} vector of structures is extended (realloc) 
to contain one more object.
\item The FITS file is opened using FITSIO
\item The required FITS keywords are read into {\tt LogInfo\_vec}.
\item MJD FITS keyword is read. This is not output, but used to sort
the files into order for output. The code exists in the source to 
calculate MJD from the UT field, but is currently commented out. It may
have to be replaced if not all instruments calculate MJD for the headers.
The code would then need to be recompiled against slalib.
\item All the L1STAT keywords are read from the header and checked
for Dp(RT) errors. Any errors are recored. See below.
\item The output file is created in the working directory, called YYYYMMDD.log.
If the file open fails, the log is still written to the screen.
\item The data are sorted by MJD and output to screen and output file
if it is open.
\end{itemize}



\section{Contents of the Log}
The log contents are listed here. For further details regarding
the meaning of any particular FITS header keyword, see `Liverpool
Telescope Fits Keyword Specification'. From left to right across
the log they cover telescope configuration, instrument configuration,
derived data and administrative logging.

\begin{tabular}{ll}
Object Name	& First 12 characters of {\tt OBJECT} from FITS header \\
Observation Type& {\tt OBSTYPE} from FITS header \\
Right Ascension & {\tt RA} from FITS header \\
Declination	& {\tt DEC} from FITS header \\
Time at start	& Extracted from part of {\tt DATE-OBS} FITS keyword. UTC \\
Airmass at start& {\tt AIRMASS} FITS keyword \\
Instrument Name & {\tt INSTRUME} FITS keyword \\
Filters		& {\tt FILTERI1} and {\tt FILTERI2} FITS keyword \\
Grating		& T.B.D.\\
Exposure Time	& {\tt EXPTIME} FITS keyword, in seconds\\
Seeing		& {\tt L1SEEING} FITS keyword, from Dp(RT)\\
Photometricity	& {\tt L1PHOTOM} FITS keyword, from Dp(RT)\\
Sky Brightness	& {\tt L1SKYBRT} FITS keyword, from Dp(RT)\\
Filename	& Literal filename read to create this object line\\
Error Code	& See below \\
\end{tabular}


Notes:
\begin{itemize}
\item Eventually I would like to convert airmass at observation start to 
an integrated mean airmass over the exposure.
\item The contents of the Filters column is likely to be heavily
instrument dependent and FITS keywords have only been defined so far for
the RATCam. These may change with new instruments.
\end{itemize}

\section{Error Codes}
Various error codes are written into the final column. Any error
codes returned by the FITSIO library will be shown. Refer to the 
FITSIO documentation for details. FITSIO errors are positive. Any
errors generated by the autologger software are negative. Most 
of these are simply responces to errors found in FITS headers 
and relate to errors in the Dp(RT). Autologger/Dp(RT) errors
are as follows.

\begin{tabular}{ll}
-2	&Unspecified error in Dp(RT) overscan subtraction\\
-4	&Unspecified error in Dp(RT) bias frame subtraction\\
-8	&Unspecified error in Dp(RT) overscan trimming \\
-16	&Unspecified error in Dp(RT) flat field correction\\
-32	&Unspecified error in Dp(RT) dark frame subtraction\\
\end{tabular}

If multiple errors are detected, the error codes are added together.


\end{document}
